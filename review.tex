\documentclass[a4paper,DIV=16]{scrartcl}

\usepackage[utf8]{inputenc}
\usepackage[T1]{fontenc}
\usepackage[german]{babel}

\usepackage[sfdefault]{FiraSans}
\usepackage{FiraMono}
\usepackage{url}
\usepackage{xcolor}
\definecolor{smdsblue}{RGB}{0, 69, 134}
\usepackage[color=smdsblue!25]{todonotes}

\pagestyle{empty}

\begin{document}
\title{Review Seminararbeit}
\date{Wintersemester 2017/18}

%% >> Bitte entsprechend Ihres EIGENEN Seminars ausfüllen!
%% --- 8< --- 8< --- 8< ---
\subtitle{\todo[inline]{Bitte die richtige Lehrveranstaltung einkommentieren}}
%\subtitle{Seminar Automotive Software Engineering}
%\subtitle{Seminar Avionic Software Engineering}
%\subtitle{Seminar Medical Information Science}
%\subtitle{Seminar Software Engineering für verteilte Systeme}
%% --- >8 --- >8 --- >8 ---

\author{\todo[inline]{Ihren Namen ausfüllen}}

\maketitle
\thispagestyle{empty}

\section*{Hinweise}
\begin{itemize}
\item Länge des Reviews: 2--3 Seiten (inklusive dieser Hinweise)
\item Abgabe per E-Mail an \url{christoph.etzel@informatik.uni-augsburg.de} bis \emph{31. Januar 2018, 12:00 Uhr}
\item Jede Frage in diesem Template muss beantwortet werden! Ersetzen sie dazu die vorhandenen \texttt{\textbackslash{}todo}-Befehle im Template durch Ihre Antworten.
\item Bitte geben Sie für jede bewertete Arbeit jeweils ein eigenes PDF-Dokument ab.
\item Die Qualität der von Ihnen verfassten Reviews geht in Ihre Gesamtnote für das Seminar ein.
\end{itemize}

\section*{Allgemeine Informationen}
\subsection*{Titel der zu bewertenden Arbeit}

\todo[inline]{Bitte ausfüllen}

\subsection*{Autor der zu bewertenden Arbeit}

\todo[inline]{Bitte ausfüllen}

\section*{Hauptinhalt der Arbeit}

\todo[inline]{Bitte fassen Sie kurz die Hauptaspekte der Arbeit zusammen}

\section*{Allgemeine Bewertung}

\subsection*{Stärken der Arbeit}

\todo[inline]{Bitte nennen Sie positive Aspekte der Arbeit}

\subsection*{Schwächen der Arbeit}

\todo[inline]{
  Wo sehen Sie Verbesserungspotential?

  Haben Sie nach dem Studium der Arbeit noch offene Fragen?}

\section*{Sachliche Korrektheit}

\todo[inline]{
  Prüfen Sie die Arbeit, so weit möglich, auf sachliche Korrektheit.

  Beispielsweise können Sie die Arbeit mit Quellen, insbesondere die korrekte
  und sinnerhaltende Zitierung, sowie deren Auswahl überprüfen.}

\section*{Äußere Form}

\todo[inline]{
  Bitte bewerten Sie die Arbeit in Hinsicht auf Aspekte der äußeren Form, beispielsweise:

  - Rechtschreibung und Grammatik

  - Sinnvolle Verwendung von Abbildungen
  
  - Beschriftung von Abbildung, Tabellen, u.ä.

  - Werden verwendete Abbildungen im Fließtext referenziert?
  
  - Optisches Erscheinungsbild}
\end{document}
